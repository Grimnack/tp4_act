\documentclass[a4paper,10pt]{article}
%\documentclass[a4paper,10pt]{scrartcl}

\usepackage[utf8]{inputenc}
\usepackage{algorithm}
\usepackage{algorithmic}
\title{TP - LA CLASSE NP}
\author{Matthieu Caron et Armand Bour}
\date{6 novembre}

\pdfinfo{%
  /Title    ()
  /Author   ()
  /Creator  ()
  /Producer ()
  /Subject  ()
  /Keywords ()
}

\begin{document}
\maketitle

\paragraph{Question 1}
Un certificat est un resultat qui montre qu'on répond Oui au problème,
Et qui peut être vérifié en temps polynomial.

L'idée c est de proposer une configuration qui peut être vérifiée.
Donc on donne une ``tournée'' sous forme de liste de villes, sans doublon,
et on pourra verifier en temps polynomial si la somme des distances entre chaque ville
voisine est inférieur à l.

Le certificat est une liste de villes sans doublon donc la taille du certificat est majoré
par le nombre totale de villes donc $[certificat]<=n$.



\begin{algorithm}
\caption{Certicat tournée plus petite que l}
\begin{algorithmic}
\REQUIRE $D, l, t$
\ENSURE $boolean$
\STATE $somme \leftarrow 0$
\STATE $i \leftarrow 0$
\WHILE{$i < t.length$}
  \STATE $somme \leftarrow somme + D[t[i],t[i+1]] $
  \STATE $i \leftarrow i + 1 $
\ENDWHILE
\STATE $somme \leftarrow somme + D[t[i],t[0]] $
\RETURN $somme \leq l $ 
\end{algorithmic}
\end{algorithm}

\paragraph{Question 2}

\paragraph{Question 3}
Il  y a $n!$ certificats possibles. Du coup pour enumerer tous les certificats, ICI ECRIS L ALGO EN LATEX
3.3 il faut un algorithm qui verifie pour chaque certificat généré par la fonction enumeration si il y a un
certificat correcte, si oui il y a une solution au pb. Cet algo est dans PSPACE puisqu'il génère une liste de certificats de taille exponentielle.

\end{document}
