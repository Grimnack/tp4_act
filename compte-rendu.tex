\documentclass[a4paper,10pt]{article}
%\documentclass[a4paper,10pt]{scrartcl}

\usepackage[utf8]{inputenc}
\usepackage{algorithm}

\title{TP - LA CLASSE NP}
\author{Matthieu Caron et Armand Bour}
\date{6 novembre}

\pdfinfo{%
  /Title    ()
  /Author   ()
  /Creator  ()
  /Producer ()
  /Subject  ()
  /Keywords ()
}

\begin{document}
\maketitle

\paragraph{Question 1}
Un certificat est un resultat qui montre qu'on répond Oui au problème,
Et qui peut être vérifié en temps polynomial.

L'idée c est de proposer une configuration qui peut être vérifiée.
Donc on donne une ``tournée'' sous forme de liste de villes, sans doublon,
et on pourra verifier en temps polynomial si la somme des distances entre chaque ville
voisine est inférieur à l.

Le certificat est une liste de villes sans doublon donc la taille du certificat est majoré
par le nombre totale de villes donc $[certificat]<=n$.



\end{document}
