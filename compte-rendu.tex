\documentclass[a4paper,10pt]{article}
%\documentclass[a4paper,10pt]{scrartcl}

\usepackage[utf8]{inputenc}
\usepackage{algorithm}
\usepackage{algorithmic}
\title{TP - LA CLASSE NP}
\author{Matthieu Caron et Armand Bour}
\date{6 novembre}

\pdfinfo{%
  /Title    ()
  /Author   ()
  /Creator  ()
  /Producer ()
  /Subject  ()
  /Keywords ()
}

\begin{document}
\maketitle

\paragraph{Question 1}
Un certificat est un resultat qui montre qu'on répond Oui au problème,
Et qui peut être vérifié en temps polynomial.

L'idée c est de proposer une configuration qui peut être vérifiée.
Donc on donne une ``tournée'' sous forme de liste de villes, sans doublon,
et on pourra verifier en temps polynomial si la somme des distances entre chaque ville
voisine est inférieur à l.

Le certificat est une liste de villes sans doublon donc la taille du certificat est majoré
par le nombre totale de villes donc $[certificat]<=n$.



\begin{algorithm}
\caption{Certicat tournée plus petite que l}
\begin{algorithmic}
\REQUIRE $D, l, t$
\ENSURE $boolean$
\STATE $somme \leftarrow 0$
\STATE $i \leftarrow 0$
\WHILE{$i < t.length$}
  \STATE $somme \leftarrow somme + D[t[i],t[i+1]] $
  \STATE $i \leftarrow i + 1 $
\ENDWHILE
\STATE $somme \leftarrow somme + D[t[i],t[0]] $
\RETURN $somme \leq l $ 
\end{algorithmic}
\end{algorithm}

\paragraph{Question 2}
Dans l'algorithme 2, random(i) crée un nombre aléatoire entre 0 et i-1

\begin{algorithm}
\caption{Generer certificat Aléatoire}
\begin{algorithmic}
\REQUIRE $n$
\ENSURE $certicat$
\STATE $certicat \leftarrow [0:n-1] $
\FOR{$i=n-1~downto~1$}
  \STATE $index \leftarrow random(i)$
  \STATE $tmp \leftarrow certicat[i] $
  \STATE $certicat[i]  \leftarrow certicat[index] $
  \STATE $certicat[index] \leftarrow tmp $
\ENDFOR
\RETURN $certicat$
\end{algorithmic}
\end{algorithm}


\paragraph{Question 3}
Il  y a $n!$ certificats possibles. Du coup pour enumerer tous les certificats,

\begin{algorithm}
\caption{enumeration}
\begin{algorithmic}
\REQUIRE $listeVilles$
\ENSURE $listeCertificats$
\IF{$len(listeVilles)\leq 1$}
  \RETURN $listeVilles$
\ENDIF
\STATE $listeCertificats \leftarrow [~]$
\FORALL{$ville~in~listeVilles$}
  \STATE $sousListes \leftarrow enumeration(listeVilles - ville)$
  \FORALL {$sousListe~in~sousListes $}
    \STATE $listeCertificats.append(ajouteEnTete(ville,sousListe)) $
  \ENDFOR
\ENDFOR
\RETURN $listeCertificats$
\end{algorithmic}
\end{algorithm}

3.3 il faut un algorithm qui verifie pour chaque certificat généré par la fonction enumeration si il y a un
certificat correcte, si oui il y a une solution au pb. Cet algo est dans PSPACE puisqu'il génère une liste de certificats de taille exponentielle.

\paragraph{Question 4}
\subparagraph{4.1}
une instance du probleme d' HamiltonCycle se reduit polynomialement dans TSP puisqu'on peut transformer un probleme hamiltonien en un probleme TPS,
nombre de sommet devient nombre de villes, la matrice reste la meme, et le chemin hamiltonien devient le certicat.

\subparagraph{4.3}
Si il existe une réduction polynomiale de Hamilton Cycle vers TSP, ça veut dire que TSP est plus dur que Hamilton Cycle.

\subparagraph{4.4}
Je ne pense pas que l'inverse soit possible car TSP est plus complexe, en effet chaque arc du graphe possède un poids, ce qui ne peut pas facilement 
être transformé... 

\paragraphe{Question 5}
\subparagraph{5.1}
Il est possible de réduire l'instance de HamiltonPath dans HamiltonCycle puisque c est la meme instance du problème, simplement les tests sont plus strict.

\subparagraph{5.3}
Si on peut transformer un probleme HamiltonPath en un probleme HamiltonCycle et un probleme HamiltonCycle en un problème TSP, alors de bout en bout on peut 
transformer un problème HamiltonPath en un problème TSP.

\paragraph{Question 7}
On peut réduire un problème TSP en un problème TSPOpt1, ce qui signifie que TSPOpt1 est plus dur que TSP. Si TSPOpt1 était dans P,
alors TSP serait dans P (ainsi HamiltonCycle dans P, hamiltonPath dans P etc à vous le million de dollar).

\paragraph{Question 8}
On peut réduire TSP dans TSPOpt1 en cherchant le minimum avec la solution polynomiale pour TSP, ainsi on peut dire que TSP aussi dur que TSPOpt1.
\end{document}
